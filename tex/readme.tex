\documentclass[a4paper,oneside]{memoir}

\usepackage[utf8]{inputenc}
\usepackage[T1]{fontenc}

\usepackage{lmodern}

%\usepackage[czech]{babel}
\usepackage{amsmath,amssymb,mathtools,amsthm,bm}

\usepackage{xspace}


\usepackage{lipsum}

\usepackage{hyperref}
\hypersetup{
    colorlinks,
    citecolor=black,
    filecolor=black,
    linkcolor=black,
    urlcolor=black
}

\title{TFGP readme}

\author{Tomáš Křen}

\hyphenation{vě-dec-ká}

\begin{document}

\theoremstyle{plain} 
\newtheorem{theorem}{Theorem} 
\newtheorem{proposition}{Proposition} 
\newtheorem{lemma}{Lemma} 
\newtheorem{preLemma}{Pre-Lemma} 
\newtheorem*{corollary}{Corollary}

\theoremstyle{definition} 
\newtheorem*{definition}{Definition} 
\newtheorem*{preDefinition}{Pre-Definition} 
\newtheorem{conjecture}{Conjecture}
\newtheorem*{example}{Example} 

\theoremstyle{remark} 
\newtheorem*{remark}{Remark} 
\newtheorem*{note}{Note} 
\newtheorem{case}{Case}

\frontmatter
\mainmatter
\maketitle

%\renewcommand{\chaptername}{Akt}

\tableofcontents*
%\clearpage

\newcommand{\red}[1]{{\color{red} #1}}


\chapter{Generating}

\section{General notions}

\begin{definition}
A $\mathit{term:type}$ statement $\mathit{M}:\mathit{\tau}$ states that (program) term $M$ has type $\tau$.   
A \textit{declaration} is a statement $s : \tau$ where $s$ is a term symbol and $\tau$ is a type.
Often we will write $s : \tau_s$ to emphasize that $\tau_s$ is the type of symbol $s$ in the supposed context.
A \textit{context} is set of declarations with distinct term symbols.\footnote{Interestingly, the definition of a \textit{context} and definition of a \textit{substitution} are almost the same. The difference is that "keys" in a context are term symbols/variables, whereas substitution "keys" are type variables. Maybe this fact could be utilized in an interesting way...}
\end{definition}

\newcommand{\then}{\Rightarrow}
\newcommand{\E}[2]{(\exists #1)\ #2}
\newcommand{\A}[2]{(\forall #1)\ #2}
\newcommand{\Ain}[3]{(\forall #1 \in #2)\ #3}


\newcommand{\op}{\operatorname}

\newcommand{\ar}{\rightarrow}
\newcommand{\ap}[2]{(#1\,#2)}
\newcommand{\defi}{\coloneqq}
\newcommand{\defe}{\mathrel{\vcentcolon\equiv}}

\newcommand{\binRule}[3]{\dfrac{#1\ ,\ #2}{#3}}
\newcommand{\triRule}[4]{\dfrac{#1\ ,\ #2\ , \ #3}{#4}}
\newcommand{\isSub}[1]{#1\ \mathit{substitution}}
\newcommand{\MGU}[2]{\op{MGU}(#1,#2)}

\newcommand{\subAx}{\textit{SUB-AX}\xspace}
\newcommand{\mguMp}{\textit{MGU-MP}\xspace}
\newcommand{\abs}[1]{\lvert #1 \rvert}

Suppose we have a context $\Gamma$. Let us consider $\mathit{term:type}$ statements derivable from the following rules (let us call them \subAx and \mguMp):

~

$\binRule{(s,\tau_s) \in \Gamma}{\isSub{\sigma}}{s : \sigma(\tau_s)}$
~~~
$\triRule{F : \tau_1 \ar \tau_2}{X : \tau^\prime_1}{\sigma = \MGU{\tau_1}{\tau^\prime_1}}{\ap{F}{X} : \sigma(\tau_2)}$


\begin{definition}
Let $M$ be a term. Term size $\abs{M}$ is the number of symbols in $M$; e.g. $\abs{\ap{f}{\ap{\ap{g}{h}}{f}}} = 4$. 
\end{definition}

\section{Reusable generating}

\newcommand{\inhab}[1]{\op{I}(#1)}
\newcommand{\sigmaPr}{\sigma^\prime}
\newcommand{\tauPr}{\tau^\prime}
\newcommand{\xPr}{x^\prime}

\newcommand{\tord}{\preccurlyeq}
\newcommand{\stord}{\prec}
\newcommand{\ordt}{\tord_\tau}
\newcommand{\tek}{\sim}
\newcommand{\ntek}{\nsim}
\newcommand{\ekt}{\tek_\tau}
\newcommand{\nekt}{\ntek_\tau}
\newcommand{\nsucct}{\nsucc_\tau}


\newcommand{\MGI}[1]{\op{MGI}_{#1}}
\newcommand{\MGIt}{\MGI{\tau}}
\newcommand{\It}{\op{I}_{\tau}}

\newcommand{\ids}{\sigma_{\op{id}}}


\begin{align*}
\op{I}_\tau(\sigma)  &\defe \E{M}{M : \sigma(\tau)}  \\
\tauPr \tord  \tau   &\defe \E{\sigma}{\tauPr = \sigma(\tau)} \\
\tauPr \stord \tau   &\defe \tauPr \tord \tau \wedge \neg (\tau \tord \tauPr) \\
\tauPr \nsucc \tau &\ \equiv \tauPr \tord \tau \vee  \neg ( \tau \tord \tauPr)   \\   
\MGI{\tau}(\sigma) &\defe \op{I}_\tau(\sigma) \wedge \Ain{\sigmaPr}{\op{I}_\tau}{\sigmaPr(\tau) \nsucc \sigma(\tau)}\\
\\
\op{I}_\tau &\defi \{ \sigma \mid \op{I}_\tau(\sigma)  \}  \\
 &\ = \{ \sigma \mid \E{M}{M : \sigma(\tau)}  \} \\
\MGI{\tau} &\defi \{ \sigma \mid \op{MGI}_\tau(\sigma)  \}  \\
 &\ = \{ \sigma \in \op{I}_\tau \mid \Ain{\sigmaPr}{\op{I}_\tau}{\sigmaPr \nsucct \sigma}  \}\\
\\
\tauPr   \tek         \tau     &\defe   \tauPr \tord \tau \wedge \tau \tord \tauPr \\
\sigmaPr \square_\tau \sigma   &\defe   \sigmaPr(\tau)\ \square\ \sigma(\tau)\text{ , e.g.:}\\
\sigmaPr \ordt   \sigma  &\defe  \sigmaPr(\tau) \tord \sigma(\tau)\\
\sigmaPr \nsucct \sigma  &\defe  \sigmaPr(\tau) \nsucc \sigma(\tau)\\
\sigmaPr \ekt    \sigma  &\defe  \sigmaPr(\tau) \tek \sigma(\tau) \\
\end{align*}

I is for Inhabitator. MG is for Most General. 



\begin{lemma}
Let $\sigma \in \MGIt$, $\sigmaPr \in \It$ such that $\sigma \ordt \sigmaPr$,
then $\sigmaPr \ekt \sigma$.
\end{lemma}
\begin{proof}
$\sigmaPr \nsucct \sigma$, since $\sigmaPr(\tau) \nsucct \sigma(\tau)$, from definition of $\MGIt$.\\
Thus $\sigmaPr \ordt \sigma \vee \neg( \sigma \ordt \sigmaPr )$.
Thus $\sigmaPr \ordt \sigma$.
Therefore $\sigmaPr \ekt \sigma$.
\end{proof}

\begin{lemma}

Let $\sigma_1, \sigma_2 \in \MGIt$ such that $\sigma_1 \nekt \sigma_2$. \\
Then $\neg (\sigma_1 \ordt \sigma_2) \wedge \neg (\sigma_2 \ordt \sigma_1)$.  

\end{lemma}
\begin{proof}
\red{todo ale mam na papiře}
\end{proof}




\begin{definition}
A substitution $\rho$ is called a renaming, if it is a permutation on the set of all variables.
\end{definition}

\begin{lemma}
$\tauPr \tek \tau \Leftrightarrow$ $\E{\text{ renaming } \rho}{\tauPr = \rho(\tau)}$.
\end{lemma}
\begin{proof}
From lemmas \ref{lem:ren1} and \ref{lem:ren2}.
\end{proof}


\begin{conjecture}
Let $\sigma_1, \sigma_2 \in \MGIt$ such that $\sigma_1 \nekt \sigma_2$. \\
Then $\A{M_1,M_2}{ M_1 : \sigma_1(\tau) \wedge M_2 : \sigma_2(\tau) \then M_1 \neq M_2}$.  
\end{conjecture}
\begin{proof}
\red{TODO}
\end{proof}



~

~

\section{Notes}

\begin{lemma}
\label{lem:ren1}
If $\tauPr \tek \tau$, then there is a \textit{renaming} $\rho$ such that $\tauPr = \rho(\tau)$.
\end{lemma}
\begin{proof}
\red{Roughly:}
$\tauPr \tord \tau \wedge \tau \tord \tauPr$, thus $\E{\rho_1, \rho_2}{\tauPr = \rho_1(\tau), \tau = \rho_2(\tauPr)}$.
$\tauPr = \rho_1(\rho_2(\tauPr))$, $\tau = \rho_2(\rho_1(\tau))$.
Take $\rho_a$ restriction of $\rho_1$ on vars of $\tau$,
Take $\rho_b$ restriction of $\rho_2$ on vars of $\tauPr$.
Thus $\rho_a \circ \rho_b = \rho_b \circ \rho_a = \ids$ 
\red{Todo pořádně..}
\end{proof}

\begin{lemma}
\label{lem:ren2}
If there is a \textit{renaming} $\rho$ such that $\tauPr = \rho(\tau)$, then $\tauPr \tek \tau$.
\end{lemma}
\begin{proof}
From $\tauPr = \rho(\tau)$ follows $\tauPr \tord \tau$. We show that $\tau \tord \tauPr$ by proving $\tau = \rho^{-1}(\tauPr)$. Since $\rho$ is renaming it is a permutation, therefore $\rho$ has inverse and $\rho$ is injective. $\rho \circ \rho^{-1} = \ids$, thus $\rho(\rho^{-1}(\tauPr)) = \tauPr = \rho(\tau)$. From injectivity we have $\rho^{-1}(\tauPr) = \tau$.
\end{proof}

In general the Most General set can be stated as:
$$\op{MG}(X,\leq) \defi \{ x \in X \mid \Ain{\xPr}{X}{\xPr \leq x \vee \neg(x \leq \xPr)} \}$$


\begin{definition}
$\ids = \{\}$ is the \textit{identity} substitution. 
\end{definition}
\begin{remark}
$\sigma \ordt \ids$ for any $\sigma$, since $\sigma(\tau) \tord \tau$ via $\sigma$, because $\sigma(\tau) = \sigma(\tau)$.
\end{remark}

~

More detailed analysis of $\nsucc$:
\begin{align*}
\tauPr \nsucc \tau &\equiv \neg (\tau \stord \tauPr)   \\
  &\equiv \neg (\tau \tord \tauPr \wedge \neg (\tauPr \tord \tau) )   \\
  &\equiv \neg (\tau \tord \tauPr) \vee \tauPr \tord \tau   \\  
  &\equiv \tauPr \tord \tau \vee  \neg (\tau \tord \tauPr)    \\
  &\equiv \tau \tord \tauPr \then \tauPr \tord \tau   \\  
  &\equiv \tau \tord \tauPr \then \tauPr \tek \tau   \\ 
  &\equiv \tauPr \succcurlyeq \tau \then \tauPr \tek \tau   \\ 
\end{align*}




~

~

\section{Probably abandoned stuff}

\newcommand{\subs}[2]{\op{subs}_{#2}(#1)}

\begin{preDefinition}
$ \subs{\tau_g}{n} \defi \{ \sigma_g \mid \E{M}{M : \sigma_g(\tau_g)}, \abs{M} = n \}$
\end{preDefinition}

Problem s touhle starou definicí: asi zahrnuje i zbytečně konkrétní substituce.

Kdybych měl $\Gamma = \{id : \alpha \ar \alpha \}$

Tak $\{\alpha \mapsto Int, \beta \mapsto  Int\} \in \subs{\alpha \ar \beta}{1}$

Čili chceme něco jako Most General Subs který pokrejvá všecky Mka.. 



~










$ \op{weakSubs}(\tau_g) \defi \{ \sigma_g \mid \inhab{\sigma_g(\tau_g)} \}$


\begin{preLemma} 
$ \subs{\tau_g}{1} = \{ \sigma_g \mid (s,\tau_s) \in \Gamma, \sigma_g = \MGU{\tau_g}{\tau^\mathit{fresh}_s}  \} $
\end{preLemma} 




\backmatter
\end{document}

